\begin{table}[H]
\centering
\label{amzparkngtxrendaporocupacao}
\scalebox{0.70}{
\begin{threeparttable}
\caption{Taxas de crescimento de rendimentos médio por tipo de ocupação entre 2012 e 2019}
\begin{tabular}{l*{2}{r}}
\midrule \midrule
                    &Tx. Cresc. (\%)\\
\hline
Oficiais maquinistas em navegação&      890,00\\
Dirigentes de serviços de saúde&      619,91\\
Entrevistadores de pesquisas de mercado&      599,29\\
Cartógrafos e agrimensores&      479,88\\
Capitães, oficiais de coberta e práticos&      345,87\\
Trabalhadores da aquicultura&      299,48\\
Bibliotecários, documentaristas e afins&      213,86\\
Químicos           &      203,20\\
Atletas e esportistas&      198,34\\
Técnicos agropecuários&      197,77\\
Geólogos e geofísicos&      189,82\\
Profissionais da proteção do meio ambiente&      183,21\\
Oficiais de polícia militar&      158,47\\
Programadores de aplicações&      155,46\\
Dirigentes de pesquisa e desenvolvimento&      142,15\\
Gerentes de centros esportivos,  de diversão e culturais&      139,30\\
Fonoaudiólogos e logopedistas&      136,19\\
Agentes de seguros  &      132,86\\
Trabalhadores qualificados do tratamento de couros e peles&      128,20\\
Técnicos em operações de tecnologia da informação e das comunicações&      126,84\\
\bottomrule
\end{tabular}
\begin{tablenotes}
\item \scriptsize{Fonte: com base nos dados da PNAD Contínua, IBGE}
\end{tablenotes}
\end{threeparttable}
}
\end{table}

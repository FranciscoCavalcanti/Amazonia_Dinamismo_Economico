\begin{table}[H]
\centering
\label{amzjovemrkngtxrendaporocupacao}
\scalebox{0.70}{
\begin{threeparttable}
\caption{Taxas de crescimento de rendimentos médio por tipo de ocupação entre 2012 e 2019}
\begin{tabular}{l*{2}{r}}
\midrule \midrule
                    &Tx. Cresc. (\%)\\
\hline
Oficiais maquinistas em navegação&      890,00\\
Cartógrafos e agrimensores&      242,44\\
Técnicos agropecuários&      204,06\\
Capitães, oficiais de coberta e práticos&      144,74\\
Tradutores, intérpretes e linguistas&      124,72\\
Inspetores de polícia e detetives&      118,41\\
Chefes de cozinha   &       99,70\\
Secretários executivos e administrativos&       86,37\\
Despachantes aduaneiros&       77,06\\
Outros professores de idiomas&       74,17\\
Instaladores e reparadores de linhas elétricas&       69,76\\
Outros professores de artes&       69,63\\
Dirigentes de produção agropecuária e silvicultura&       67,72\\
assistentes de medicina&       67,19\\
Assessores financeiros e em investimentos&       66,59\\
Diretores gerais e gerentes gerais&       61,73\\
Instaladores de material isolante térmico e acústico&       58,88\\
Atletas e esportistas&       56,94\\
Mineiros e operadores de máquinas e de instalações em minas e pedreiras&       56,54\\
Trabalhadores elementares da jardinagem e horticultura&       56,00\\
\bottomrule
\end{tabular}
\begin{tablenotes}
\item \scriptsize{Fonte: com base nos dados da PNAD Contínua, IBGE}
\end{tablenotes}
\end{threeparttable}
}
\end{table}

\begin{table}[H]
\centering
\label{amzparkngtxocuporocupacao}
\scalebox{0.70}{
\begin{threeparttable}
\caption{Taxas de crescimento de ocupações por tipo de ocupação entre 2012 e 2019}
\begin{tabular}{l*{2}{r}}
\midrule \midrule
                    &Tx. Cresc. (\%)\\
\hline
Alfaiates, modistas, chapeleiros e peleteiros&    5.945,23\\
Ajudantes de professores&    1.290,11\\
Trabalhadores e assistentes sociais de nível médio&      841,14\\
Dentistas auxiliares e ajudantes de odontologia&      475,86\\
Instaladores e reparadores em tecnologias da informação e comunicações&      401,57\\
Oficiais das forças armadas&      386,64\\
Dietistas e nutricionistas&      362,40\\
Médicos especialistas&      345,86\\
Balconistas dos serviços de alimentação&      344,34\\
Dirigentes financeiros&      310,19\\
Dirigentes de serviços de saúde&      305,04\\
Impressores         &      303,07\\
Arquitetos de edificações&      279,10\\
Analistas de sistemas&      265,80\\
Operadores de máquinas de lavar, tingir e passar roupas&      260,82\\
Trabalhadores de cuidados pessoais a domicílios&      243,92\\
Ajudantes de ambulâncias&      240,13\\
Técnicos em aparelhos de diagnóstico e tratamento médico&      236,07\\
Trabalhadores da degustação e classificação de alimentos e bebidas&      234,74\\
Profissionais de relações públicas&      211,77\\
\bottomrule
\end{tabular}
\begin{tablenotes}
\item \scriptsize{Fonte: com base nos dados da PNAD Contínua, IBGE}
\end{tablenotes}
\end{threeparttable}
}
\end{table}

\begin{table}[H]
\centering
\label{amzmtrkngtxocuporocupacao}
\scalebox{0.70}{
\begin{threeparttable}
\caption{Taxas de crescimento de ocupações por tipo de ocupação entre 2012 e 2019}
\begin{tabular}{l*{2}{r}}
\midrule \midrule
                    &Tx. Cresc. (\%)\\
\hline
Alfaiates, modistas, chapeleiros e peleteiros&    2.097,20\\
Trabalhadores de cuidados pessoais a domicílios&      740,67\\
Instaladores de material isolante térmico e acústico&      530,40\\
Vendedores a domicilio&      480,41\\
Trabalhadores e assistentes sociais de nível médio&      438,30\\
Ajudantes de professores&      418,08\\
Instrutores de educação física e atividades recreativas&      416,33\\
Engenheiros eletricistas&      410,98\\
Dentistas auxiliares e ajudantes de odontologia&      405,82\\
Cuidadores de animais&      319,11\\
Dirigentes financeiros&      310,67\\
Técnicos e assistentes fisioterapeutas&      269,28\\
Construtores de casas&      267,25\\
Dirigentes de políticas e planejamento&      267,04\\
Profissionais de vendas técnicas e médicas (exclusive tic)&      266,73\\
Arquitetos de edificações&      257,12\\
Profissionais da proteção do meio ambiente&      254,11\\
Vendedores ambulantes de serviços de alimentação&      252,78\\
Analistas financeiros&      240,09\\
Comerciantes de lojas&      236,67\\
\bottomrule
\end{tabular}
\begin{tablenotes}
\item \scriptsize{Fonte: com base nos dados da PNAD Contínua, IBGE}
\end{tablenotes}
\end{threeparttable}
}
\end{table}

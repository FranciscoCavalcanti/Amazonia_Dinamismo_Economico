\begin{table}[H]
\centering
\label{amzmtrkngtxocuporocupacao}
\scalebox{0.60}{
\begin{threeparttable}
\caption{Taxas de crescimento de ocupações por tipo de ocupação entre 2012 e 2019}
\begin{tabular}{l*{2}{r}}
\midrule \midrule
                    &Taxa de crescimento (\%)\\
\hline
Alfaiates, modistas, chapeleiros e peleteiros&    2.097,20\\
Outras ocupações elementares não classificadas anteriormente&    1.943,09\\
Trabalhadores elementares da indústria de transformação não classificados anteriormente&      798,00\\
Trabalhadores de cuidados pessoais a domicílios&      740,67\\
Instaladores de material isolante térmico e acústico&      530,40\\
Marinheiros de coberta e afins&      493,43\\
Vendedores a domicilio&      480,41\\
Dirigentes de explorações de mineração&      475,56\\
Profissionais em direito não classificados anteriormente&      448,68\\
Trabalhadores e assistentes sociais de nível médio&      438,30\\
Ajudantes de professores&      418,08\\
Instrutores de educação física e atividades recreativas&      416,33\\
Engenheiros eletricistas&      410,98\\
Dentistas auxiliares e ajudantes de odontologia&      405,82\\
Vendedores não classificados anteriormente&      401,92\\
Cuidadores de animais&      319,11\\
Dirigentes financeiros&      310,67\\
Bombeiros           &      279,07\\
Técnicos e assistentes fisioterapeutas&      269,28\\
Construtores de casas&      267,25\\
\bottomrule
\end{tabular}
\begin{tablenotes}
\item \scriptsize{Fonte: com base nos dados da PNAD Contínua, IBGE}
\end{tablenotes}
\end{threeparttable}
}
\end{table}

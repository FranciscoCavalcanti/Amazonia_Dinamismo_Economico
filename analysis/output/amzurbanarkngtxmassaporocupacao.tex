\begin{table}[H]
\centering
\label{amzurbanarkngtxmassaporocupacao}
\scalebox{0.70}{
\begin{threeparttable}
\caption{Taxas de crescimento de massa de rendimentos por tipo de ocupação entre 2012 e 2019}
\begin{tabular}{l*{2}{r}}
\midrule \midrule
                    &Tx. Cresc. (\%)\\
\hline
Alfaiates, modistas, chapeleiros e peleteiros&    3.811,39\\
Vendedores por telefone&      651,26\\
Ajudantes de professores&      639,47\\
Dentistas auxiliares e ajudantes de odontologia&      612,11\\
Dirigentes de serviços de saúde&      602,85\\
Trabalhadores e assistentes sociais de nível médio&      540,82\\
Oficiais maquinistas em navegação&      525,96\\
Construtores de casas&      462,63\\
Cuidadores de animais&      390,25\\
Dirigentes financeiros&      302,53\\
Entrevistadores de pesquisas de mercado&      274,33\\
Trabalhadores de cuidados pessoais a domicílios&      272,59\\
Balconistas dos serviços de alimentação&      271,69\\
Profissionais da proteção do meio ambiente&      268,36\\
Instaladores e reparadores em tecnologias da informação e comunicações&      265,05\\
Atletas e esportistas&      256,18\\
Gerentes de centros esportivos,  de diversão e culturais&      255,86\\
Vendedores a domicilio&      255,14\\
Instaladores de material isolante térmico e acústico&      232,81\\
Vendedores ambulantes de serviços de alimentação&      218,93\\
\bottomrule
\end{tabular}
\begin{tablenotes}
\item \scriptsize{Fonte: com base nos dados da PNAD Contínua, IBGE}
\end{tablenotes}
\end{threeparttable}
}
\end{table}

\begin{table}[H]
\centering
\label{amzjovemrkngnocuporocupacao}
\scalebox{0.60}{
\begin{threeparttable}
\caption{Variação absoluta do número de ocupações por tipo de ocupação entre 2012 e 2019}
\begin{tabular}{l*{2}{r}}
\midrule \midrule
                    &  Variação\\
\hline
Vendedores a domicilio&      47.184\\
Comerciantes de lojas&      43.148\\
Balconistas dos serviços de alimentação&      18.371\\
Mecânicos e reparadores de veículos a motor&      15.125\\
Especialistas em tratamento de beleza e afins&      14.914\\
Repositores de prateleiras&      14.590\\
Outras ocupações elementares não classificadas anteriormente&      13.731\\
Vendedores ambulantes de serviços de alimentação&      13.608\\
Vendedores não classificados anteriormente&      12.299\\
Trabalhadores elementares da indústria de transformação não classificados anteriormente&      12.013\\
Cabeleireiros       &      11.284\\
Vendedores de quiosques e postos de mercados&       9.665\\
Criadores de gado e trabalhadores qualificados da criação de gado&       9.580\\
Ajudantes de professores&       8.892\\
Profissionais de ensino não classificados anteriormente&       8.452\\
Condutores de motocicletas&       8.096\\
Padeiros, confeiteiros e afins&       7.889\\
Escriturários gerais&       7.288\\
Cuidadores de crianças&       6.968\\
Trabalhadores de cuidados pessoais a domicílios&       6.846\\
\bottomrule
\end{tabular}
\begin{tablenotes}
\item \scriptsize{Fonte: com base nos dados da PNAD Contínua, IBGE}
\end{tablenotes}
\end{threeparttable}
}
\end{table}

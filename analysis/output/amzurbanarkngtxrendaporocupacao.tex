\begin{table}[H]
\centering
\label{amzurbanarkngtxrendaporocupacao}
\scalebox{0.60}{
\begin{threeparttable}
\caption{Taxas de crescimento de rendimentos médio por tipo de ocupação entre 2012 e 2019}
\begin{tabular}{l*{2}{r}}
\midrule \midrule
                    &Taxa de crescimento (\%)\\
\hline
Confeccionadores e afinadores de instrumentos musicais&    2.945,85\\
Oficiais maquinistas em navegação&      890,00\\
Entrevistadores de pesquisas de mercado&      174,52\\
Capitães, oficiais de coberta e práticos&      157,34\\
Governantas e mordomos domésticos&      147,65\\
Meteorologistas     &      140,51\\
Trabalhadores da aquicultura&      139,67\\
Dirigentes de serviços de saúde&      133,04\\
Profissionais de nível médio de medicina tradicional e alternativa&      129,29\\
Limpadores de fachadas&      122,41\\
Profissionais de vendas de tecnologia da informação e comunicações&      121,23\\
Gerentes de centros esportivos,  de diversão e culturais&      112,60\\
Técnicos em documentação sanitária&      108,08\\
Instrutores em tecnologias da informação&      102,06\\
Apicultores, sericicultores e trabalhadores qualificados da apicultura e sericicultura&      101,08\\
Oficiais de polícia militar&       98,51\\
Especialistas em formação de pessoal&       93,12\\
Técnicos agropecuários&       92,17\\
Oficiais de bombeiro militar&       90,68\\
Trabalhadores do serviço de pessoal&       90,50\\
\bottomrule
\end{tabular}
\begin{tablenotes}
\item \scriptsize{Fonte: com base nos dados da PNAD Contínua, IBGE}
\end{tablenotes}
\end{threeparttable}
}
\end{table}

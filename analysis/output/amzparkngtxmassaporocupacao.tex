\begin{table}[H]
\centering
\label{amzparkngtxmassaporocupacao}
\scalebox{0.70}{
\begin{threeparttable}
\caption{Taxas de crescimento de massa de rendimentos por tipo de ocupação entre 2012 e 2019}
\begin{tabular}{l*{2}{r}}
\midrule \midrule
                    &Tx. Cresc. (\%)\\
\hline
Dirigentes de serviços de saúde&    3.377,52\\
Trabalhadores e assistentes sociais de nível médio&      764,25\\
Médicos especialistas&      673,79\\
Oficiais das forças armadas&      593,73\\
Técnicos em aparelhos de diagnóstico e tratamento médico&      583,12\\
Oficiais maquinistas em navegação&      525,96\\
Dirigentes de pesquisa e desenvolvimento&      468,28\\
Dietistas e nutricionistas&      458,81\\
Técnicos em operações de tecnologia da informação e das comunicações&      419,14\\
Impressores         &      384,01\\
Operadores de máquinas de lavar, tingir e passar roupas&      381,36\\
Profissionais da proteção do meio ambiente&      361,78\\
Vendedores por telefone&      357,06\\
Dirigentes financeiros&      344,26\\
Profissionais de nível médio de serviços estatísticos, matemáticos e afins&      280,24\\
Balconistas dos serviços de alimentação&      268,71\\
Analistas de sistemas&      262,24\\
Bibliotecários, documentaristas e afins&      258,42\\
Instaladores e reparadores em tecnologias da informação e comunicações&      246,63\\
Policiais           &      232,81\\
\bottomrule
\end{tabular}
\begin{tablenotes}
\item \scriptsize{Fonte: com base nos dados da PNAD Contínua, IBGE}
\end{tablenotes}
\end{threeparttable}
}
\end{table}

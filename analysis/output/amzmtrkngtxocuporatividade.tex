\begin{table}[H]
\centering
\label{amzmtrkngtxocuporatividade}
\scalebox{0.70}{
\begin{threeparttable}
\caption{Taxas de crescimento de ocupações por grupamentos de atividade entre 2012 e 2019}
\begin{tabular}{l*{2}{r}}
\midrule \midrule
                    &Tx. Cresc. (\%)\\
\hline
Seleção, agenciamento e locação de mão-de-obra&      603,62\\
Cultivo de soja     &      318,51\\
Cultivo de banana   &      287,40\\
Pré-escola e ensino fundamental&      216,17\\
Serviços ambulantes de alimentação&      214,17\\
Condomínios prediais&      212,28\\
Comércio de tecidos, artefatos de tecidos e armarinho&      180,01\\
Fabricação de cabines, carrocerias, reboques e peças para veículos automotores&      158,66\\
Comércio de matérias-primas agrícolas e animais vivos&      155,05\\
Atividades de condicionamento físico&      148,81\\
Armazenamento, carga e descarga&      141,99\\
Atividades paisagísticas&      141,45\\
Confecção, sob medida, de artigos do vestuário&      140,52\\
Criação de aves   &      138,92\\
Cultivo de milho    &      138,51\\
Transporte ferroviário e metroferroviário&      130,69\\
Atividades auxiliares dos transportes e atividades relacionadas à organização do transporte de carga&      128,33\\
Outras atividades de serviços pessoais&      118,28\\
Metalurgia dos metais não-ferrosos&      117,19\\
Serviços de limpeza e de apoio a edifícios, exceto condomínios prediais&      115,23\\
\bottomrule
\end{tabular}
\begin{tablenotes}
\item \scriptsize{Fonte: com base nos dados da PNAD Contínua, IBGE}
\end{tablenotes}
\end{threeparttable}
}
\end{table}

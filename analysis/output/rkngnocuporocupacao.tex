\begin{table}[H]
\centering
\label{rkngnocuporocupacao}
\scalebox{0.70}{
\begin{threeparttable}
\caption{Variação absoluta do número de ocupações por tipo de ocupação entre 2012 e 2019}
\begin{tabular}{l*{2}{r}}
\midrule \midrule
                    &  Variação\\
\hline
Comerciantes de lojas&     294.423\\
Vendedores a domicilio&     193.082\\
Criadores de gado e trabalhadores qualificados da criação de gado&     128.960\\
Escriturários gerais&      93.179\\
Alfaiates, modistas, chapeleiros e peleteiros&      69.902\\
Vendedores ambulantes de serviços de alimentação&      59.959\\
Vendedores de quiosques e postos de mercados&      57.871\\
Cozinheiros         &      52.757\\
Trabalhadores de limpeza de interior de edifícios, escritórios, hotéis e outros estabelecimentos&      51.008\\
Mecânicos e reparadores de veículos a motor&      48.003\\
Profissionais de nível médio de enfermagem&      37.773\\
Condutores de automóveis, taxis e caminhonetes&      34.558\\
Especialistas em tratamento de beleza e afins&      33.617\\
Balconistas dos serviços de alimentação&      31.365\\
Agricultores e trabalhadores qualificados no cultivo de hortas, viveiros e jardins&      31.135\\
Padeiros, confeiteiros e afins&      30.796\\
Trabalhadores de cuidados pessoais a domicílios&      27.888\\
Professores do ensino pré-escolar&      27.683\\
Advogados e juristas&      24.233\\
Cabeleireiros       &      24.190\\
\bottomrule
\end{tabular}
\begin{tablenotes}
\item \scriptsize{Fonte: com base nos dados da PNAD Contínua, IBGE}
\end{tablenotes}
\end{threeparttable}
}
\end{table}

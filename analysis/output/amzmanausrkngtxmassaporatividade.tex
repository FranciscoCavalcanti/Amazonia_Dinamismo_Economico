\begin{table}[H]
\centering
\label{amzmanausrkngtxmassaporatividade}
\scalebox{0.70}{
\begin{threeparttable}
\caption{Taxas de crescimento de massa de rendimentos por grupamentos de atividade entre 2012 e 2019}
\begin{tabular}{l*{2}{r}}
\midrule \midrule
                    &Tx. Cresc. (\%)\\
\hline
Metalurgia dos metais não-ferrosos&      952,02\\
Fabricação e refino do açúcar&      750,87\\
Confecção, sob medida, de artigos do vestuário&      591,09\\
Serviços de assistência social sem alojamento&      539,92\\
Atividades artísticas, criativas e de espetáculos&      397,57\\
Cultivo de frutas cítricas&      311,25\\
Extração de pedras, areia e argila&      307,27\\
Seleção, agenciamento e locação de mão-de-obra&      299,37\\
Fundição          &      274,69\\
Criação de aves   &      271,21\\
Outros serviços coletivos prestados pela administração pública - Municipal&      234,11\\
Fabricação de embalagens e de produtos diversos de papel, cartolina, papel-cartão e papelão ondulado&      231,12\\
Atividades auxiliares dos seguros, da previdência complementar e dos planos de saúde&      222,92\\
Comércio de equipamentos e produtos de tecnologias de informação e comunicação&      222,39\\
Atividades de organizações sindicais&      190,06\\
Fabricação de vidro e produtos de vidro&      188,96\\
Fabricação de artefatos para pesca e esporte e de brinquedos e jogos recreativos&      183,22\\
Atividades veterinárias&      168,47\\
Manutenção e reparação de máquinas e equipamentos&      162,31\\
Produção florestal&      160,77\\
\bottomrule
\end{tabular}
\begin{tablenotes}
\item \scriptsize{Fonte: com base nos dados da PNAD Contínua, IBGE}
\end{tablenotes}
\end{threeparttable}
}
\end{table}

\begin{table}[H]
\centering
\label{amzmtrkngtxmassaporatividade}
\scalebox{0.70}{
\begin{threeparttable}
\caption{Taxas de crescimento de massa de rendimentos por grupamentos de atividade entre 2012 e 2019}
\begin{tabular}{l*{2}{r}}
\midrule \midrule
                    &Tx. Cresc. (\%)\\
\hline
Seleção, agenciamento e locação de mão-de-obra&      971,74\\
Fabricação de embalagens e de produtos diversos de papel, cartolina, papel-cartão e papelão ondulado&      611,28\\
Fabricação de cabines, carrocerias, reboques e peças para veículos automotores&      478,13\\
Aqüicultura        &      443,83\\
Produção de sementes e mudas certificadas&      382,85\\
Cultivo de soja     &      344,22\\
Atividades de condicionamento físico&      334,53\\
Pré-escola e ensino fundamental&      295,84\\
Metalurgia dos metais não-ferrosos&      294,46\\
Cultivo de banana   &      283,05\\
Comércio de matérias-primas agrícolas e animais vivos&      267,34\\
Atividades de organizações associativas patronais, empresariais e profissionais&      227,26\\
Extração de carvão mineral&      219,35\\
Criação de aves   &      217,97\\
Condomínios prediais&      217,75\\
Serviços ambulantes de alimentação&      213,93\\
Cultivo de milho    &      206,02\\
Comércio de tecidos, artefatos de tecidos e armarinho&      179,42\\
Atividades de profissionais da área de saúde, exceto médicos e odontólogos&      169,81\\
Atividades paisagísticas&      165,80\\
\bottomrule
\end{tabular}
\begin{tablenotes}
\item \scriptsize{Fonte: com base nos dados da PNAD Contínua, IBGE}
\end{tablenotes}
\end{threeparttable}
}
\end{table}

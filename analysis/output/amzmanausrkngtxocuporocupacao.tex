\begin{table}[H]
\centering
\label{amzmanausrkngtxocuporocupacao}
\scalebox{0.60}{
\begin{threeparttable}
\caption{Taxas de crescimento de ocupações por tipo de ocupação entre 2012 e 2019}
\begin{tabular}{l*{2}{r}}
\midrule \midrule
                    &Taxa de crescimento (\%)\\
\hline
Comerciantes de lojas&    1.671,87\\
Outras ocupações elementares não classificadas anteriormente&    1.597,59\\
Instaladores e reparadores em tecnologias da informação e comunicações&      906,52\\
Agricultores e trabalhadores qualificados no cultivo de hortas, viveiros e jardins&      801,22\\
Farmacêuticos      &      747,57\\
Vendedores por telefone&      740,25\\
Vendedores a domicilio&      709,86\\
Artesãos de tecidos, couros e materiais semelhantes&      585,06\\
Vendedores não classificados anteriormente&      471,73\\
Gerentes de hotéis &      418,44\\
Engenheiros de meio ambiente&      405,30\\
Operadores de máquinas para elaborar alimentos e produtos afins&      399,28\\
Psicólogos         &      394,88\\
Profissionais de relações públicas&      351,37\\
Veterinários       &      313,00\\
Reguladores e operadores de máquinas-ferramentas&      296,65\\
Dentistas auxiliares e ajudantes de odontologia&      293,15\\
Outros professores de artes&      288,06\\
Coletores de dinheiro em máquinas automáticas de venda e leitores de medidores&      284,04\\
Arquitetos de edificações&      281,17\\
\bottomrule
\end{tabular}
\begin{tablenotes}
\item \scriptsize{Fonte: com base nos dados da PNAD Contínua, IBGE}
\end{tablenotes}
\end{threeparttable}
}
\end{table}

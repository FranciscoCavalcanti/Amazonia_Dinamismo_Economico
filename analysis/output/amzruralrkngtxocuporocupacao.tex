\begin{table}[H]
\centering
\label{amzruralrkngtxocuporocupacao}
\scalebox{0.60}{
\begin{threeparttable}
\caption{Taxas de crescimento de ocupações por tipo de ocupação entre 2012 e 2019}
\begin{tabular}{l*{2}{r}}
\midrule \midrule
                    &Taxa de crescimento (\%)\\
\hline
Alfaiates, modistas, chapeleiros e peleteiros&    9.284,80\\
Vendedores não classificados anteriormente&    2.895,91\\
Ajudantes de professores&    2.551,05\\
Agricultores e trabalhadores qualificados de cultivos mistos&    1.311,42\\
Outros trabalhadores qualificados e operários da construção não classificados anteriormente&    1.286,95\\
Engenheiros eletricistas&      803,45\\
Trabalhadores de cuidados pessoais a domicílios&      701,02\\
Outros professores de idiomas&      669,64\\
Graduados e praças da polícia militar&      592,42\\
Técnicos em aparelhos de diagnóstico e tratamento médico&      539,57\\
Dirigentes de pesquisa e desenvolvimento&      427,17\\
Técnicos em engenharia de minas e metalurgia&      416,11\\
Marinheiros de coberta e afins&      401,03\\
Professores de universidades e do ensino superior&      360,68\\
Atletas e esportistas&      355,77\\
Cuidadores de animais&      355,27\\
Instaladores e reparadores em tecnologias da informação e comunicações&      333,56\\
Graduados e praças do corpo de bombeiros&      278,76\\
Balconistas dos serviços de alimentação&      276,00\\
Vendedores a domicilio&      269,03\\
\bottomrule
\end{tabular}
\begin{tablenotes}
\item \scriptsize{Fonte: com base nos dados da PNAD Contínua, IBGE}
\end{tablenotes}
\end{threeparttable}
}
\end{table}
